%
% File ACL2016.tex
%

\documentclass[11pt]{article}
\usepackage{acl2016}
\usepackage{times}
\usepackage{latexsym}

%\aclfinalcopy % Uncomment this line for the final submission
%\def\aclpaperid{***} %  Enter the acl Paper ID here

% To expand the titlebox for more authors, uncomment
% below and set accordingly.
% \addtolength\titlebox{.5in}    

\newcommand\BibTeX{B{\sc ib}\TeX}


\title{Inferring Topic Domains from Topics in Newspaper and Web Data}

% Author information can be set in various styles:
% For several authors from the same institution:
% \author{Author 1 \and ... \and Author n \\
%         Address line \\ ... \\ Address line}
% if the names do not fit well on one line use
%         Author 1 \\ {\bf Author 2} \\ ... \\ {\bf Author n} \\
% For authors from different institutions:
% \author{Author 1 \\ Address line \\  ... \\ Address line
%         \And  ... \And
%         Author n \\ Address line \\ ... \\ Address line}
% To start a seperate ``row'' of authors use \AND, as in
% \author{Author 1 \\ Address line \\  ... \\ Address line
%         \AND
%         Author 2 \\ Address line \\ ... \\ Address line \And
%         Author 3 \\ Address line \\ ... \\ Address line}
% If the title and author information does not fit in the area allocated,
% place \setlength\titlebox{<new height>} right after
% at the top, where <new height> can be something larger than 2.25in
\author{Roland Schäfer\\
	    Freie Universität Berlin\\
	    Habelschwerdter Allee 45\\
	    14196 Berlin, Germany\\
	    {\tt roland.schaefer@fu-berlin.de}
	  \And
	Felix Bildhauer\\
  	Institut für Deutsche Sprache\\
  	R5, 6--13\\
  	68161 Mannheim, Germany\\
  {\tt bildhauer@ids-mannheim.de}}

\date{}

\begin{document}

\maketitle

\begin{abstract}
\end{abstract}

\section{Introduction}

Why topic domain?

Automatic meta data: desirable not JUST for web data.

Short comments on register scene and poor results in recent Globbe paper.

Mention corpus comparison as important field: Kilgarriff, WCC, "Biemann et al."

\section{Gold standard Data}

Corpora \cite{KupietzEa2010}, SchäBi 2012, Schä 2015

Annotation scheme: Sharoff; mention that it was developed in repeated annotation processes based on annotator feedback; mention that design goal was roughly 10 to 20 topic domains

\section{Experiment Setup}

Pre-processing

Algorithms (LSI/LDA)

Gold and plus versions

Filters and lexicon thresholds

Numbers of topics

SVM vs. Trees; SVM kernel selection

Elimination of very small topic domains

\section{Results}

Accuracy and $\kappa$ for Cow, Dereko, Coreko and plus variants

4 plots

\section{Conclusions and Outlook}

The results presented here are preliminary, but highly encouraging (over 90\% accuracy on training data and 70\% accuracy in cross-validation on some data sets), and they indicate the route to be taken in further experiments.
First of all, there appears to be a solid connection between induced topic distributions and externally defined topic domains.
The relative poor performance in cross-validation experiments indicates that larger gold standard data sets are required.
Such data sets are currently being annotated.
Secondly, there appears to be a significant difference in the topic distribution and the topic\slash domain mapping in newspaper and web corpora.
This is indicated by the drop in classification accuracy when newspaper and web data are pooled.
As such, 
In future experiments, it remains to be discovered whether larger corpora can alleviate this divergence, finally enabling us to decide whether separate models are required or joint models can be trained.
Thirdly, the highly skewed topic distributions in both newspaper and web data sets as well as comments from annotators indicate that splitting up some topic domains might lead to a better fit.

\bibliography{coreko}
\bibliographystyle{acl2016}

\end{document}
